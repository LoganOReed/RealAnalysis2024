%! TEX root = **/000-main.tex
% vim: spell spelllang=en:

\begin{problem}{1}
  Prove that if $E$ is a Lebesgue measurable subset of $\mathbb{R}^{n}$ with $\left| E \right| < \infty$ then for all $\varepsilon > 0$ there exists a compact set $K$ such that $K \subset E$ and $\left| E - K \right| < \varepsilon$.
\end{problem}

\begin{solution}
  Let $\varepsilon > 0$ and suppose that $E$ is bounded.
  Then there exists some (closed and bounded) box $I$ such that $\left| I \cap E \right| = |E|$.
  Since $I$ and $E$ are measurable, $I - E$ is measurable and so there exists some open $V \supset I - E$ such that $\left| V \right| < \left| I - E \right| + \varepsilon$.
  Note that $I - V$ is compact as $I$ is compact and $V$ is open.
  From this and the previous statement we have 
  \[
    \begin{aligned}
     \left| I - E \right| &> \left| V \right| - \varepsilon \\
     \left| I \right| &> \left| V \right| + \left| I \cap E \right| - \varepsilon \\
     \left| I - V \right| &> \left| I \cap E \right| - \varepsilon     
    \end{aligned}
  .\] 
  Using this and $\left| I \cap E \right| = \left| E \right|$ we obtain 
  \[
  \left| I - V \right| > \left| I \cap E \right| - \varepsilon = \left| E \right| - \varepsilon
  .\] 
  Hence $\left| E - (I - V) \right| < \varepsilon$, and since $I - V$ is compact we have shown the desired result.

  Now we will consider the case where $E$ is not bounded.
  Let $\left\{ I_{i} \right\}_{i=1}^{\infty}$ such that $I_{i} \subset I_{i+1}$ which approaches $\mathbb{R}^{n}$, and define $E_{i} = E \cap I_{i}$.
  Since $E = \bigcup_{i=1}^{\infty} E_{i} $ and $E_{i} \subset E_{i+1}$, we have $\lim_{i \to \infty} \left| E_{i} \right| = \left| E \right|$.
  So we can pick $N$ such that
  \[
  \left| E \right| \leq \left| E_{N} \right| + \frac{\varepsilon}{2}
  .\] 
  Further, since $E_{N}$ is bounded we can apply the previous case to find some compact set $K \subset E_{N}$ where
  \[
  \left| E_{N} \right| \leq \left| K \right| + \frac{\varepsilon}{2}
  .\] 
  From the previous two inequalities we have that $\left| E \right| \leq \left| K \right| + \varepsilon$.
  

\end{solution}

\pagebreak

\begin{problem}{2}
  Let $E \subset [0,1]$ be a measurable set with $\left| E \right| = 1$. Prove that $E$ is dense in $[0,1]$.
\end{problem}



\begin{solution}
  Suppose, for the sake of contradiction, that $A \subset [0,1]$ is a nonempty open set such that $E \cap A = \emptyset$.
  Let  $a,b \in [0,1]$ such that $a < b$ and $(a,b) \subset A$.
  Since $\left| (a,b) \right| > 0$, we have $\left| A \right|>0$ by monotonicity.
  Then we can construct the inequality 
  \[
    1 = \left| E \right| < \left| E \right| + \left| A \right| = \left| E \cup A \right| \leq \left| [0,1] \right| = 1
  .\] 
  Since $1 \not< 1$, we have obtained a contradiction.
  Thus $E$ is dense in $[0,1]$.
\end{solution}

\pagebreak

\begin{problem}{3}
  Let $0 < \varepsilon < 1$ and let $\left\{ q_{n} \right\}_{n=1}^{\infty}$ be an enumeration of all rational numbers in $[0,1]$.
  Let 
  \[
    E = \bigcup_{n=1}^{\infty}\left[q_{n} - \varepsilon 2^{-(n+1)}, q_{n} + \varepsilon 2^{-(n+1)}\right]
  \] 
  and set $F = E \cap [0,1]$. Prove that
  \begin{enumerate}
    \item $F$ is dense in $[0,1]$
    \item $F$ is measurable and $0 < \left| F \right| \leq \varepsilon$, conclude that $\left| [0,1] - F \right| \geq 1 - \varepsilon$ 
    \item $F$ is not closed.
  \end{enumerate}

\end{problem}


\begin{solution}
  \todo{Solve 3}
\end{solution}

\pagebreak

\begin{problem}{4}
  Let $E,F \subset \mathbb{R}^{n}$ be bounded sets and consider the symmetric difference $E \Delta F = \left( E - F \right) \cup \left( F - E \right) $.
  Prove that
  \[
    \left| m^{*}(E) - m^{*}(F) \right| \leq m^{*}(E \Delta F)
  .\] 
\end{problem}


\begin{solution}
  \todo{Solve 4}
\end{solution}

\pagebreak

\begin{problem}{5}
  Suppose that $E$ is a given set, and $\mathcal{O}_{n}$ is the open set
  \[
  \mathcal{O}_{n} = \left\{ x : d(x,E) < \frac{1}{n} \right\}
  .\] 
  Show that if E is compact, then $|E| = \lim_{n \to \infty} \left| \mathcal{O}_{n} \right|$.
\end{problem}

\begin{solution}
  Let $E$ be compact. 
  Then $E$ is bounded, and so $\mathcal{O}_{1}$ is bounded.
  Since $\mathcal{O}_{1}$ is bounded, $\left| \mathcal{O}_{1} \right| < \infty$.
  Finally, since $\bigcap_{n=1}^{\infty}\mathcal{O}_{n} = E$ and $\mathcal{O}_{1} \supset \mathcal{O}_{2} \supset \ldots $ we have
  \[
  \left| E \right| = \left| \bigcap_{n=1}^{\infty} \mathcal{O}_{n} \right| = \lim_{n \to \infty} \left| \mathcal{O}_{n} \right|
  .\] 

\end{solution}

\pagebreak

\begin{problem}{6}
  Let $\left\{ A_{j} \right\}_{j=1}^{\infty}$ be a disjoint sequence of measurable sets.
  Prove that for any set $A \subset \mathbb{R}^{n}$ we have 
  \[
  m^{*}\left( \bigcup_{j=1}^{N} A_{j} \cap A  \right) = \sum_{j=1}^{N} m^{*}\left( A_{j} \cap A \right)
  \] 
  for each $N$.
\end{problem}

\begin{solution}
  By outer measure subadditivity we have 
  \[
  m^{*}\left( \bigcup_{j=1}^{N}  A_{j} \cap A \right) \leq \sum_{j=1}^{\infty} m^{*}\left( A_{j} \cap A \right)
  .\] 
  Define $B_{n} = \bigcup_{n=1}^{n}A_{n} $.
  Consider the base case when $n = 1$.
  Obviously $m^{*}\left( \bigcup_{j=1}^{n} A_{j} \cap A  \right) = \sum_{j=1}^{n} m^{*}\left( A_{j} \cap A \right)$
  Then, since all $A_{j}$ are measurable we have the following for any $n \in \mathbb{N}$
  \[
  m^{*}(A \cap B_{n}) = m^{*}\left( \left( A \cap B_{n} \right) \cap A_{n} \right) + m^{*}\left( \left( A \cap B_{n} \right) \cap A_{n}^{c} \right) = m^{*}\left( A \cap A_{n} \right) + m^{*}\left( A \cap B_{n-1} \right)
  .\] 
  Thus, by induction on $n$, 
  \[
    m^{*}\left( \bigcup_{j=1}^{n}  A_{j} \cap A \right) = \sum_{j=1}^{n} m^{*}\left( A_{j} \cap A \right)
  .\] 
  Then, by monotonicity, we have 
  \[
  m^{*}\left( \bigcup_{j=1}^{\infty} A_{j} \cap A  \right) \geq m^{*}\left( \bigcup_{j=1}^{n} A_{j} \cap A  \right) = \sum_{j=1}^{n} m^{*}\left( A_{j} \cap A \right)
  .\] 
  Finally, since we've shown inequality in both directions, we have
  \[
    m^{*}\left( \bigcup_{j=1}^{N} A_{j} \cap A  \right) = \sum_{j=1}^{N} m^{*}\left( A_{j} \cap A \right)
  .\] 

  
\end{solution}
