%! TEX root = **/000-main.tex
% vim: spell spelllang=en:

\begin{problem}{1}
  Let $A,B \subset \mathbb{R}^{n}$ with $A$ compact and $B$ closed. Prove that if $A \cap B = \emptyset$, then the distance $d(A,B) > 0$.
  Show an example of $A,B$ both closed with $A \cap B = \emptyset$ and $d(A,B) = 0$.
\end{problem} 

\begin{solution}

  Let $A \cap B = \emptyset$, and define the function $ \delta_{B} : A \to \mathbb{R}^{} $ such that $\delta_{B}(x) = d(x,B)$ for $x \in A$.
  Since $\delta_{B}$ is continuous in $x$ and $A$ is compact, $\delta_{B}$ is compact.
  From this we can guarantee $\inf_{}\delta_{B}(A) \in \delta_{B}(A)$.
  If $0 \in \delta_{B}(A)$ then there must exist some $a \in A$ and $b \in B$ such that $d(a,b) = 0$.
  However, from the definition of a metric, this only occurs when $a = b$.
  This contradicts the previous statement that $A \cap B = \emptyset$, and so $ \inf_{}\delta_{B}(A) > 0$.
  Thus $d(A,B) > 0$.

  For the example, let $A = \mathbb{N}$ and $B = \left\{ n + \frac{1}{2n} : n \in \mathbb{N} \right\}$.
  Since $2n > 1$, $B$ will never be an integer and so $A \cap B = \emptyset$.
  Also, since $(n+1) + \frac{1}{2(n+1)}$ is monotone increasing and $\lim_{n\to \infty} ((n+1) + \frac{1}{2(n+1)}) - (n + \frac{1}{2n}) = 1$, we have that $A,B$ are closed and disjoint.
  Finally, we have
  \[
  d(A,B) = \inf\{\left| a - b \right| : a \in A, b \in B\} = \inf\left| \frac{1}{2n} \right| = 0
  .\] 

\end{solution}

\begin{problem}{2}
  Let $ f : [a,b] \to \mathbb{R}^{} $ be continuous. Prove the graph of $f, G(f) = \left\{ (x,f(x)) : x \in [a,b] \right\}$ is a set of measure zero in $\mathbb{R}^{2}$.
\end{problem}
    
\begin{solution}
  Let $G(f) = \left\{ (x,f(x)) : x \in [a,b] \right\}$.
  Take $\varepsilon > 0$, and since $f$ is uniformly continuous there exists some $\delta \in \mathbb{R}^{}$ such that $\left| x - y \right| < \delta$ implies $\left| f(x) - f(y) \right|<\varepsilon$.
  Also, let $n \in \mathbb{N}$ such that $\frac{b-a}{n} < \delta$.
  Then $G(f)$ is contained within 
  \[
    \bigcup_{i=0}^{n-1} \left[a + \frac{(b-a)i}{n}, a + \frac{(b-a)(i+1)}{n} \right] \times \left[ f\left( \frac{(b-a)i}{n} \right) - \varepsilon, f\left( \frac{(b-a)i}{n}\right)  + \varepsilon \right]
  .\] 
  Using sub-additivity, we have
  \begin{equation*}
    \begin{split}
      &m\left( \bigcup_{i=0}^{n-1} \left[a + \frac{(b-a)i}{n}, a + \frac{(b-a)(i+1)}{n} \right] \times \left[ f\left( \frac{(b-a)i}{n} \right) - \varepsilon, f\left( \frac{(b-a)i}{n}\right)  + \varepsilon \right] \right) \\ 
      &\leq \sum_{i=0}^{n-1} m\left(  \left[a + \frac{(b-a)i}{n}, a + \frac{(b-a)(i+1)}{n} \right] \times \left[ f\left( \frac{(b-a)i}{n} \right) - \varepsilon, f\left( \frac{(b-a)i}{n}\right)  + \varepsilon \right] \right)\\
      &= n \frac{2\varepsilon(b-a)}{n} = 2(b-a)\varepsilon.
    \end{split}
  \end{equation*}
  Finally, taking $n \to \infty$ gives us $\left| G(f) \right| = 0$.

\end{solution}

\pagebreak

\begin{problem}{3}
Let $A \subset \mathbb{R}^{d}$ be measurable with $|A| < \infty$.

(a) Show that the function $f(r) = \left| A \cap \prod_{i=1}^{d } [-\frac{r}{2}, \frac{r}{2}] \right|$ is nondecreasing, bounded, and continuous.

(b) Show there exists a Lebesgue measurable set $B \subset A$ such that $\left| B \right| = \frac{1}{\sqrt{2} } \left| A \right|$.
\end{problem}

\begin{solution}
  Let $f(r) = \left| A \cap \prod_{i=1}^{d } [-\frac{r}{2}, \frac{r}{2}] \right|$.
  When $r\leq s$, $\left| \prod_{i=1}^{d } [-\frac{r}{2}, \frac{r}{2}] \right| \subset \left| \prod_{i=1}^{d } [-\frac{s}{2}, \frac{s}{2}] \right|$ and so the monotonicity of measures directly implies that $f$ is nondecreasing.
  Also, since $\left| A \right| < \infty$, there exists some $M \in \mathbb{R}^{}$ such that $\left| A \right| < M$.
  Then for sufficiently large $r$, $\left| f(r) \right| < M$, and so $f$ is bounded.
  Next, fix $\varepsilon > 0$ and let $\delta = \frac{\varepsilon}{d}$.
  Suppose $\left| x - y \right| < \delta$ and WLOG let $x \leq  y$, then
    \begin{align*}
      \left| f(x) - f(y) \right| &= \left| \left| A \cap \prod_{i=1}^{d } [-\frac{x}{2}, \frac{x}{2}] \right| - \left| A \cap \prod_{i=1}^{d } [-\frac{y}{2}, \frac{y}{2}] \right| \right| \\
                                 &= \left| A \cap \left( \prod_{i=1}^{d } [-\frac{x}{2}, \frac{x}{2}] - \prod_{i=1}^{d } [-\frac{y}{2}, \frac{y}{2}] \right) \right| \tag{Class Corollary}\\
                                 &= \left| A \cap \left( \prod_{i=1}^{d } [-\frac{y}{2}, -\frac{x}{2}] \cup  [\frac{x}{2}, \frac{y}{2}] \right) \right| \\
                                 &\leq \left| \left( \prod_{i=1}^{d } [-\frac{y}{2}, -\frac{x}{2}] \cup  [\frac{x}{2}, \frac{y}{2}] \right) \right| \tag{Monotonicity} \\
                                 &< d \left( \frac{\delta}{2} + \frac{\delta}{2} \right) = \varepsilon 
    .\end{align*}
  Thus $f$ is continuous.

  For the second part of the problem, we begin by noting that $f(0) = 0$ and \\$\lim_{r \to \infty}f(r) = \left| A \right| $.
  Then, since $f$ is continuous, the Intermediate Value Theorem gives us the existence of some $x$ such that $\left| B \right| = f(x) = \frac{1}{\sqrt{2} }\left| A \right|$.
  Lastly, since $B$ is defined as the intersection of $A$ with other sets, $B \subset A$.

  

\end{solution}

\begin{problem}{4}
Suppose that $E \subset \mathbb{R}^{}$ is Lebesgue measurable and $F = \cup_{x \in E}^{} \left[ x-1,x+1 \right]$ (a union of closed intervals).
Prove that $F$ is Lebesgue measurable.
\end{problem}

\begin{solution}
  Leveraging the fact that the measure is invariant under translation, we have
  \[
    \bigcup_{x \in E}[x-1,x+1] = \bigcup_{x \in E} (x-1,x+1) \cup (E - 1) \cup (E+1)
  .\] 
  Since all but the last two terms are open, their union is open and thus measurable.
  As mentioned previously, the last two terms are translations of a measurable set and so they are measurable.
  Finally, the countable union of measurable sets are measurable, thus $F$ is measurable.
\end{solution}

\begin{problem}{5}
Let $A$ be the subset of $\left[ 0,1 \right]$ which consists of all numbers which do not have digit $4$ appearing in their decimal expansion.
Find $\left| A \right|$.
\end{problem}

\begin{solution}
  Define the function $\# : [0,1] \to \mathbb{N} $ such that the first $4$ occurs at the $(\#(x)+1)$-th digit for $x \in [0,1]$, and let $A_{k} = \left\{ x\in[0,1] : \#(x) = k \right\}$.
  Then we have $A_{i+1} \subseteq A_{i}$ and $A = \lim_{n \to \infty} A_{n} = \bigcap_{n=1}^{ \infty } A_{n} $.
  So, from the class notes, $m(A) = \lim_{n \to \infty} m(A_{n})$.
  Now, let us define the set $T(X) = \left\{ x \in X : \#(x) \neq  0  \right\}$. 
  Obviously $A_{0} = T([0,1])$, and we can see that $A_{n+1} = 10^{-(n+1)}T(10^{n+1}A_{n})$.
  This recursive relation can be seen as bit shifting forward until first place values in the set are the same as $[0,1]$, followed by a shift back to store the data.
  Also note that $10^{n+1}A_{n}$ is a union of finite intervals whose bounds are in $\mathbb{N}$.
  This gives us
  \[
  m(T(10^{n+1}A_{n})) = 10^{n+1} \frac{9}{10} m(A_{n}) = 9(10^{n})m(A_{n}) \\
  \] 
  and in turn we have
  \[
  m(A_{n+1}) = \frac{9}{10}m(A_{n}))
  .\] 
  Finally, $m(A) = \lim_{n \to \infty} m(A_{n}) = 0$.
  \todo{Go over this and give more detail to last three lines}

\end{solution}


\begin{problem}{6}
Show that $E \subset \mathbb{R}^{n}$ is measurable if and only if 
\[
\left| A \right|_{e } = \left| A \cap E  \right|_{e} + \left| A - E  \right|_{e}
\] 
for every subset $A$ of $\mathbb{R}^{n}$.
\end{problem}

\begin{solution}
  \todo{Solve 6}
\end{solution}
