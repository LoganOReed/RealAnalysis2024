%! TEX root = **/000-main.tex
% vim: spell spelllang=en:

\begin{problem}{1}
  Let $\mu$ be a Borel measure in $\mathbb{R}^{}$ with $\mu\left( \mathbb{R}^{} \right) < \infty$.
  Define $f\left( x \right) = \mu\left( - \infty, x  \right]$ for $x \in \mathbb{R}^{}$.
  Prove that

  (a) $f$ is monotone increasing

  (b) $\mu \left( a,b \right] = f\left( b \right) - f\left( a \right)$

  (c) $f$ is continuous from the right

  (d) $\lim_{x \to -\infty} f\left( x  \right) = 0$.
\end{problem}

\begin{solution}
  Let $f(r) = \left| A \cap \prod_{i=1}^{d } [-\frac{r}{2}, \frac{r}{2}] \right|$.
  When $r\leq s$, $\left| \prod_{i=1}^{d } [-\frac{r}{2}, \frac{r}{2}] \right| \subset \left| \prod_{i=1}^{d } [-\frac{s}{2}, \frac{s}{2}] \right|$ and so the monotonicity of measures directly implies that $f$ is nondecreasing.
  Also, since $\left| A \right| < \infty$, there exists some $M \in \mathbb{R}^{}$ such that $\left| A \right| < M$.
  Then for sufficiently large $r$, $\left| f(r) \right| < M$, and so $f$ is bounded.
  Next, fix $\varepsilon > 0$ and let $\delta = \frac{\varepsilon}{d}$.
  Suppose $\left| x - y \right| < \delta$ and WLOG let $x \leq  y$, then
    \begin{align*}
      \left| f(x) - f(y) \right| &= \left| \left| A \cap \prod_{i=1}^{d } [-\frac{x}{2}, \frac{x}{2}] \right| - \left| A \cap \prod_{i=1}^{d } [-\frac{y}{2}, \frac{y}{2}] \right| \right| \\
                                 &= \left| A \cap \left( \prod_{i=1}^{d } [-\frac{x}{2}, \frac{x}{2}] - \prod_{i=1}^{d } [-\frac{y}{2}, \frac{y}{2}] \right) \right| \tag{Class Corollary}\\
                                 &= \left| A \cap \left( \prod_{i=1}^{d } [-\frac{y}{2}, -\frac{x}{2}] \cup  [\frac{x}{2}, \frac{y}{2}] \right) \right| \\
                                 &\leq \left| \left( \prod_{i=1}^{d } [-\frac{y}{2}, -\frac{x}{2}] \cup  [\frac{x}{2}, \frac{y}{2}] \right) \right| \tag{Monotonicity} \\
                                 &< d \left( \frac{\delta}{2} + \frac{\delta}{2} \right) = \varepsilon 
    .\end{align*}
  Thus $f$ is continuous.

  For the second part of the problem, we begin by noting that $f(0) = 0$ and \\$\lim_{r \to \infty}f(r) = \left| A \right| $.
  Then, since $f$ is continuous, the Intermediate Value Theorem gives us the existence of some $x$ such that $\left| B \right| = f(x) = \frac{1}{\sqrt{2} }\left| A \right|$.
  Lastly, since $B$ is defined as the intersection of $A$ with other sets, $B \subset A$.
\end{solution}

\pagebreak

\begin{problem}{2}
  Let $\left( S, \Sigma, \mu \right)$ be a finite measure space and let $f : S \to \mathbb{R}^{} $ be a nonnegative $\Sigma$-measurable function.
  Prove that $\lim_{k \to \infty} \int_{S} \! f^{k} \, \mathrm{d}\mu $ exists if and only if \\
  $\mu\left( \left\{ x \in S : f(x) > 1 \right\} \right) = 0$.
\end{problem}
    
\begin{solution}
  Let $G(f) = \left\{ (x,f(x)) : x \in [a,b] \right\}$.
  Take $\varepsilon > 0$, and since $f$ is uniformly continuous there exists some $\delta \in \mathbb{R}^{}$ such that $\left| x - y \right| < \delta$ implies $\left| f(x) - f(y) \right|<\varepsilon$.
  Also, let $n \in \mathbb{N}$ such that $\frac{b-a}{n} < \delta$.
  Then $G(f)$ is contained within 
  \[
    \bigcup_{i=0}^{n-1} \left[a + \frac{(b-a)i}{n}, a + \frac{(b-a)(i+1)}{n} \right] \times \left[ f\left( \frac{(b-a)i}{n} \right) - \varepsilon, f\left( \frac{(b-a)i}{n}\right)  + \varepsilon \right]
  .\] 
  Using sub-additivity, we have
  \begin{equation*}
    \begin{split}
      &m\left( \bigcup_{i=0}^{n-1} \left[a + \frac{(b-a)i}{n}, a + \frac{(b-a)(i+1)}{n} \right] \times \left[ f\left( \frac{(b-a)i}{n} \right) - \varepsilon, f\left( \frac{(b-a)i}{n}\right)  + \varepsilon \right] \right) \\ 
      &\leq \sum_{i=0}^{n-1} m\left(  \left[a + \frac{(b-a)i}{n}, a + \frac{(b-a)(i+1)}{n} \right] \times \left[ f\left( \frac{(b-a)i}{n} \right) - \varepsilon, f\left( \frac{(b-a)i}{n}\right)  + \varepsilon \right] \right)\\
      &= n \frac{2\varepsilon(b-a)}{n} = 2(b-a)\varepsilon.
    \end{split}
  \end{equation*}
  Finally, taking $n \to \infty$ gives us $\left| G(f) \right| = 0$.
\end{solution}

\pagebreak

\begin{problem}{3}
  Let $f \in L\left( X,d \mu \right)$. Prove that $\left| \int_{X} \! f \, \mathrm{d}\mu  \right| = \int_{X} \! \left| f \right| \, \mathrm{d}\mu $ if and only if $f = \left| f \right|$ a.e. or $f = - \left| f \right|$ a.e.
\end{problem}

\begin{solution}
  For the forward direction, suppose $\left| \int_{X} \! f \, \mathrm{d}\mu  \right| = \int_{X} \! \left| f \right| \, \mathrm{d}\mu $.
  Using this we can obtain
  \[
  \left| \int_{\left\{ f > 0 \right\}} \! \left| f \right| \, \mathrm{d} \mu + \int_{\left\{ f < 0 \right\}} \! - \left| f \right| \, \mathrm{d} \mu  \right| = \left| \int_{X} \! f \, \mathrm{d} \mu  \right| = \int_{X} \! \left| f \right| \, \mathrm{d} \mu =  \int_{\left\{ f > 0 \right\}} \! \left| f \right| \, \mathrm{d} \mu + \int_{\left\{ f < 0 \right\}} \! - \left| f \right| \, \mathrm{d} \mu
  .\] 
  This is a specific case of the triangle inequality, so we can use the fact that for any $a,b \in \mathbb{R}^{}$, $\left| a + b \right| = \left| a \right| + \left| b \right|$ if and only if $ab \geq 0$.
  Applying this we get
  \[
  \left( \int_{\left\{ f > 0 \right\}} \! \left| f \right| \, \mathrm{d} \mu  \right) \left( - \int_{\left\{ f < 0 \right\}} \! \left| f \right| \, \mathrm{d} \mu \right) = 0
  \] 
  where we have a strict equality since the two terms are of differing signs.
  However, strict equality only holds if one or both are zero.
  Thus $f = \left| f \right|$ a.e. or $f = - \left| f \right|$ a.e.

  For the reverse direction, suppose $f = \left| f \right|$ a.e. or $f = - \left| f \right|$ a.e.
  Then we have
  \[
  \int_{X} \! \pm \left| f \right| \, \mathrm{d} \mu = \int_{X} \! f \, \mathrm{d} \mu = \left| \int_{X} \! f \, \mathrm{d} \mu  \right|
  ,\]
  where the last equality holds as the first term is equal to its absolute value.
\end{solution}

\pagebreak

\begin{problem}{4}
  Let $f \in L\left( E, d \mu \right)$ and $E = \bigcup_{j=1}^{\infty} E_{j} $ with $E_{n} \subset E_{j+1}$ all $\mu$-measurable sets.
  Prove that
  \[
  \int_{E} \! \left| f \chi_{E_{j}} - f \right| \, \mathrm{d} \mu \to 0 \text{ as } j \to \infty
  .\] 
\end{problem}

\begin{solution}
  Leveraging the fact that the measure is invariant under translation, we have
  \[
    \bigcup_{x \in E}[x-1,x+1] = \bigcup_{x \in E} (x-1,x+1) \cup (E - 1) \cup (E+1)
  .\] 
  Since all but the last two terms are open, their union is open and thus measurable.
  As mentioned previously, the last two terms are translations of a measurable set and so they are measurable.
  Finally, the countable union of measurable sets are measurable, thus $F$ is measurable.
\end{solution}

\pagebreak

\begin{problem}{5}
  Let $\left( S, \Sigma, \mu \right)$ be a measure space.
  Suppose $\left\{ f_{k} \right\}$ is a sequence of nonnegative functions in $L\left( S, d \mu \right)$ such that $f_{k} \to f$ a.e. in $\mu$ for some $f\in L\left( S,d \mu \right)$.
  Furthermore assume $\int_{S} \! f_{k} \, \mathrm{d} \mu \to \int_{S} \! f \, \mathrm{d} \mu $ as $k \to \infty$.
  Prove that $\int_{E} \! f_{k} \, \mathrm{d} \mu \to \int_{E} \! f \, \mathrm{d} \mu  $ for each measurable set $E \subset S$.
\end{problem}

\begin{solution}
  Define the function $\# : [0,1] \to \mathbb{N} $ such that the first $4$ occurs at the $(\#(x)+1)$-th digit for $x \in [0,1]$, and let $A_{k} = \left\{ x\in[0,1] : \#(x) = k \right\}$.
  Then we have $A_{i+1} \subseteq A_{i}$ and $A = \lim_{n \to \infty} A_{n} = \bigcap_{n=1}^{ \infty } A_{n} $.
  So, from the class notes, $m(A) = \lim_{n \to \infty} m(A_{n})$.
  Now, let us define the set $T(X) = \left\{ x \in X : \#(x) \neq  0  \right\}$. 
  Obviously $A_{0} = T([0,1])$, and we can see that $A_{n+1} = 10^{-(n+1)}T(10^{n+1}A_{n})$.
  This recursive relation can be seen as bit shifting forward until first place values in the set are the same as $[0,1]$, followed by a shift back to store the data.
  Also note that $10^{n+1}A_{n}$ is a union of finite intervals whose bounds are in $\mathbb{N}$.
  This gives us
  \[
  m(T(10^{n+1}A_{n})) = 10^{n+1} \frac{9}{10} m(A_{n}) = 9(10^{n})m(A_{n}) \\
  \] 
  and in turn we have
  \[
  m(A_{n+1}) = \frac{9}{10}m(A_{n}))
  .\] 
  Finally, $m(A) = \lim_{n \to \infty} m(A_{n}) = 0$.
\end{solution}

\pagebreak

\begin{problem}{6}
  Let $\left( S, \Sigma, \mu \right)$ be a measure space and let $\phi$ be an additive set function.
  Prove that
  \[
  \phi < < \mu \text{ iff } \overline{V} < < \mu \text{ and } \underline{V} < < \mu \text{ iff } V < < \mu
  .\] 
\end{problem}

\begin{solution}
  Let $E \subset \mathbb{R}^{n}$ be measurable.
  From class we can define $H = \bigcap_{n=0}^{\infty} G_{k}$ for $G_{k}$ open such that $A \subset H$ and $\left| A \right|_{e} = \left| H \right|_{e}$.
  From subadditivity, we have $\left| A \right|_{e} \leq  \left| A \cap E \right|_{e} + \left| A - E \right|_{e}$.
  Note that $H,H\cap E,H - E$ are measurable.
  Then $\left| A \cap E \right|_{e} + \left| A - E \right|_{e} \leq \left| H \cap E \right|_{e} + \left| H - E \right|_{e} = \left| H \right|_{e} = \left| A \right|_{e}$.
  I could not figure out a proof for the converse.
\end{solution}
