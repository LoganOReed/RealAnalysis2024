%! TEX root = **/000-main.tex
% vim: spell spelllang=en:

\begin{problem}{1}
  Let $\mu$ be a Borel measure in $\mathbb{R}^{}$ with $\mu\left( \mathbb{R}^{} \right) < \infty$.
  Define $f\left( x \right) = \mu\left( - \infty, x  \right]$ for $x \in \mathbb{R}^{}$.
  Prove that

  (a) $f$ is monotone increasing

  (b) $\mu \left( a,b \right] = f\left( b \right) - f\left( a \right)$

  (c) $f$ is continuous from the right

  (d) $\lim_{x \to -\infty} f\left( x  \right) = 0$.
\end{problem}

\begin{solution}
  Throughout the proof we will use $\mu\left( a,b \right]$ as a shorthand for $\mu\left( \left( a,b \right] \right)$.

  For part (a), let $x_{1} \leq x_{2} \in \mathbb{R}^{}$.
  Note that $(-\infty, x_{1}] \subset (-\infty, x_{2}]$.
  By monotonicity of measure,
  \[
    f\left( x_{1} \right) = \mu \left( -\infty, x_{1} \right] \leq \mu \left( -\infty, x_{2} \right] = f\left( x_{2} \right)
  .\] 
  So $f$ is monotone increasing.

  For part (b), WLOG let $a < b \in \mathbb{R}^{}$.
  Since $\left( -\infty, a \right] \sqcup \left( a,b \right] = \left( -\infty,b \right]$, 
  \[
    \mu \left( \left( -\infty, a \right] \sqcup \left( a,b \right] \right) = \mu \left( -\infty, a \right] + \mu \left( a,b \right] = \mu \left( -\infty, b \right]
  .\] 
  Thus $\mu\left( a,b \right] = f\left( b \right) - f\left( a \right)$.

  TODO: Ask sean or Ali about c and d.
\end{solution}

\pagebreak

\begin{problem}{2}
  Let $\left( S, \Sigma, \mu \right)$ be a finite measure space and let $f : S \to \mathbb{R}^{} $ be a nonnegative $\Sigma$-measurable function.
  Prove that $\lim_{k \to \infty} \int_{S} \! f^{k} \, \mathrm{d}\mu $ exists if and only if \\
  $\mu\left( \left\{ x \in S : f(x) > 1 \right\} \right) = 0$.
\end{problem}
    
\begin{solution}
  For the forward direction, suppose $\lim_{k \to \infty} \int_{S} \! f^{k} \, \mathrm{d} \mu$ exists.
  We have 
  \[
    \lim_{k \to \infty} \int_{S} \! f^{k} \, \mathrm{d} \mu = \lim_{k \to \infty} \int_{S-\left\{ f > 1 \right\}} \! f^{k} \, \mathrm{d} \mu + \lim_{k \to \infty} \int_{\left\{ f > 1 \right\}} \! f^{k} \, \mathrm{d} \mu \geq \lim_{k \to \infty} \int_{\left\{ f > 1 \right\}} \! f^{k} \, \mathrm{d} \mu
  .\] 
  Assume, for the sake of contradiction, that $\left| \left\{ f > 1 \right\} \right| > 0$.
  Then, since $f$ is nonnegative, $f^{k+1} \geq f^{k}$ for any $k$ and $f^{k}$ is unbounded as $k \to \infty$ for any $x \in \left\{ f > 1 \right\}$.
  So we have
  \[
  \lim_{k \to \infty} \int_{S} \! f^{k} \, \mathrm{d} \mu \geq \infty 
  .\] 
  This contradicts $\lim_{k \to \infty} \int_{S} \! f^{k} \, \mathrm{d} \mu$ existing, thus $\left| \left\{ f > 1 \right\} \right| = 0$.

  For the reverse direction, suppose $\left| \left\{ f > 1 \right\} \right| = 0$.
  Let $g(x) = 1$, which is measurable in $S$ since $\mu$ is of finite measure.
  Similarly, $\chi_{\left\{ f = 1 \right\}} \in L\left( S, d \mu \right)$ since $\mu$ is of finite measure.
  Also, $\left| f^{k} \right| \leq g$ a.e. and $f^{k} \to \chi_{\left\{ f = 1 \right\}}$ as $k \to \infty$.
  So, by DCT,
  \[
  \lim_{k \to \infty} \int_{S} \! f^{k} \, \mathrm{d} \mu = \int_{S} \! \chi_{\left\{ f = 1 \right\}} \, \mathrm{d} \mu  
  .\] 
\end{solution}

\pagebreak

\begin{problem}{3}
  Let $f \in L\left( X,d \mu \right)$. Prove that $\left| \int_{X} \! f \, \mathrm{d}\mu  \right| = \int_{X} \! \left| f \right| \, \mathrm{d}\mu $ if and only if $f = \left| f \right|$ a.e. or $f = - \left| f \right|$ a.e.
\end{problem}

\begin{solution}
  For the forward direction, suppose $\left| \int_{X} \! f \, \mathrm{d}\mu  \right| = \int_{X} \! \left| f \right| \, \mathrm{d}\mu $.
  Using this we can obtain
  \[
  \left| \int_{\left\{ f > 0 \right\}} \! \left| f \right| \, \mathrm{d} \mu + \int_{\left\{ f < 0 \right\}} \! - \left| f \right| \, \mathrm{d} \mu  \right| = \left| \int_{X} \! f \, \mathrm{d} \mu  \right| = \int_{X} \! \left| f \right| \, \mathrm{d} \mu =  \int_{\left\{ f > 0 \right\}} \! \left| f \right| \, \mathrm{d} \mu + \int_{\left\{ f < 0 \right\}} \! - \left| f \right| \, \mathrm{d} \mu
  .\] 
  This is a specific case of the triangle inequality, so we can use the fact that for any $a,b \in \mathbb{R}^{}$, $\left| a + b \right| = \left| a \right| + \left| b \right|$ if and only if $ab \geq 0$.
  Applying this we get
  \[
  \left( \int_{\left\{ f > 0 \right\}} \! \left| f \right| \, \mathrm{d} \mu  \right) \left( - \int_{\left\{ f < 0 \right\}} \! \left| f \right| \, \mathrm{d} \mu \right) = 0
  \] 
  where we have a strict equality since the two terms are of differing signs.
  However, strict equality only holds if one or both are zero.
  Thus $f = \left| f \right|$ a.e. or $f = - \left| f \right|$ a.e.

  For the reverse direction, suppose $f = \left| f \right|$ a.e. or $f = - \left| f \right|$ a.e.
  Then we have
  \[
  \int_{X} \! \pm \left| f \right| \, \mathrm{d} \mu = \int_{X} \! f \, \mathrm{d} \mu = \left| \int_{X} \! f \, \mathrm{d} \mu  \right|
  ,\]
  where the last equality holds as the first term is equal to its absolute value.
\end{solution}

\pagebreak

\begin{problem}{4}
  Let $f \in L\left( E, d \mu \right)$ and $E = \bigcup_{j=1}^{\infty} E_{j} $ with $E_{n} \subset E_{j+1}$ all $\mu$-measurable sets.
  Prove that
  \[
  \int_{E} \! \left| f \chi_{E_{j}} - f \right| \, \mathrm{d} \mu \to 0 \text{ as } j \to \infty
  .\] 
\end{problem}

\begin{solution}
  To begin, let us note that $\left| f \chi_{E_{j}} - f \right| = \left| f \chi_{E-E_{j}} \right|$ and $\left| f \chi_{E-E_{j}} \right| \leq \left| f \right|$.
  Also, since $\left| f \right|$ is measurable we know that $\left| f \chi_{E_{j}} - f \right|$ is measurable.
  Further, since $E-E_{j} \to \emptyset$ as $j \to \infty$, $\left| f \chi_{E_{j}} - f \right| = \left| f \chi_{E - E_{j}} \right| \to 0$ as $j \to \infty$.
  Finally, by DCT we have $\int_{E} \! \left| f \chi_{E_{j}} - f \right| \, \mathrm{d} \mu \to 0 $ as $ j \to \infty$.
\end{solution}

\pagebreak

\begin{problem}{5}
  Let $\left( S, \Sigma, \mu \right)$ be a measure space.
  Suppose $\left\{ f_{k} \right\}$ is a sequence of nonnegative functions in $L\left( S, d \mu \right)$ such that $f_{k} \to f$ a.e. in $\mu$ for some $f\in L\left( S,d \mu \right)$.
  Furthermore assume $\int_{S} \! f_{k} \, \mathrm{d} \mu \to \int_{S} \! f \, \mathrm{d} \mu $ as $k \to \infty$.
  Prove that $\int_{E} \! f_{k} \, \mathrm{d} \mu \to \int_{E} \! f \, \mathrm{d} \mu  $ for each measurable set $E \subset S$.
\end{problem}

\begin{solution}
  Let $E \subset S$ be measurable.
  Since $f_{k} \to f$ as $k \to \infty$, we have that $f_{k} \chi_{E} \to f \chi_{E}$ as $k \to \infty$.
  Also, we know $\sup_{k}\left\{ f_{k} \chi_{E} \right\}$ is measurable since for any $\alpha \in \mathbb{R}^{}$, $\left( f_{k} \chi_{E} \right)^{-1}\left( \left( \alpha,\infty \right] \right)$ is measurable and so $\sup_{k}\left\{ f_{k} \chi_{E} \right\} = \bigcup_{k=1}^{\infty}  \left( f_{k} \chi_{E} \right)^{-1}\left( \left( \alpha,\infty \right] \right)$ is measurable. 
  Obviously $\left| f_{k} \chi_{E} \right| \leq \sup_{k}\left\{ f_{k} \chi_{E} \right\}$ a.e., so by DCT, as $k \to \infty$, 
  \[
  \int_{E} \! f_{k} \, \mathrm{d} \mu = \int_{S} \! f_{k} \chi_{E} \, \mathrm{d} \mu \to \int_{X} \! f \chi_{E} \, \mathrm{d} \mu = \int_{E} \! f \, \mathrm{d} \mu    
  .\] 
\end{solution}

\pagebreak

\begin{problem}{6}
  Let $\left( S, \Sigma, \mu \right)$ be a measure space and let $\phi$ be an additive set function.
  Prove that
  \[
  \phi < < \mu \text{ iff } \overline{V} < < \mu \text{ and } \underline{V} < < \mu \text{ iff } V < < \mu
  .\] 
\end{problem}

\begin{solution}
  Let $Z \in \Sigma$ such that $\mu\left( Z \right) = 0$.
  We will begin by proving the first if and only if statement.
  For the forward direction, suppose $\phi < < \mu$.
  Then, by definition, $\overline{V}\left( Z \right) = \sup_{A \subset Z, A \in \Sigma}\left\{ \phi\left( A \right) \right\} = 0$.
  Using Jordan Decomposition, we get $\underline{V}\left( Z \right) = \overline{V}\left( Z \right) - \phi\left( Z \right) = 0 - 0 = 0$.
  So, by definition, $\overline{V} < < \mu$ and $\underline{V} < < \mu$.
  For the reverse direction, suppose $\overline{V} < < \mu$ and $\underline{V} < < \mu$.
  By Jordan Decomposition, $\phi(Z) = \overline{V}\left( Z \right) - \underline{V}\left( Z \right) = 0 - 0 = 0$.
  Thus $\phi < < \mu$.
  This proves the first if and only if statement.

  Now we will prove the second if and only if statement.
  For the forward direction, suppose $\overline{V} < < \mu$ and $\underline{V} < < \mu$.
  By definition, $V(Z) = \overline{V}\left( Z \right) + \underline{V}\left( Z \right) = 0 + 0 = 0$.
  So $V < < \mu$.
  For the reverse direction, suppose $V < < \mu$.
  By definition, $V\left( Z \right) = \overline{V}\left( Z \right) + \underline{V}\left( Z \right) = 0$.
  Since $\overline{V}$ and $\underline{V}$ are both nonnegative functions, $\overline{V}\left( Z \right) = \underline{V}\left( Z \right) = 0$.
  Thus $\overline{V} < < \mu$ and $\underline{V} < < \mu$.
  This proves the second if and only if statement.
  Therefore the if and only if chain has been proven.
\end{solution}
